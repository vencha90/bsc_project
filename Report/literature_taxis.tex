\subsection{Taxis}
\label{sec:taxis}

\paragraph{} Taxis (also known as \textit{taxicabs}) are an important part of
public transportation. Because of their prevalence worldwide and importance in
transportation a wide range of literature has been produced on taxis. For this
project, the most relevant area of this literature is economical modelling of
taxi markets, an overview of which is given by
\textcite{Salanova2011taxi+review}. A major topic in the research and
discussion on taxis is taxi market regulation, and parts of it are relevant and
will be considered in some detail. Three different types of taxi markets can be
distinguished: cruising taxi market when a passenger hails the taxi no the
street, phone-order taxi market, and taxi ranks where multiple taxis wait for
passengers.

\paragraph{Taxis in the wider context of public transport} 

\subsubsection{Regulation}

\paragraph{}Regulation is a sensitive topic for taxi research as no general
consensus has been reached on whether it is recommended.
\textcite{Cairns1996taxi+competition} investigated economic workings of taxi
markets and incorporated results of earlier research in their economic
equilibria findings. They concluded that regulation is needed to achieve 
non-negative profits (the so-called economic second best).
\textcite{Oecd2007taxi+policy}, cited by \textcite{Salanova2011taxi+review}
listed arguments both for and against regulation as observed in different
countries, and noted that markets with widely varying regulation can operate
successfully. It is important to note that some markets considered
\textit{deregulated} still have some form of fare regulation, for example,
taxis in New Zealand are required to list their maximum fares based on time and
distance, but are not forced to follow them
\parencite{Gaunt1995taxi+newzealand}.

\subsubsection{Economic modelling} 

\paragraph{}Taxi market modelling was first done by
\textcite{Douglas1972taxi+regulation}, according to
\textcite{Salanova2011taxi+review}. He investigated a regulated cruising-taxi
market (where a customer hails a taxi on the street on visual contact) and
defined the fundamental taxi problem to be finding an equilibrum of an optimal
level of service matching an optimal price. His limited model has been used as
reference by all the later authors cited by \textcite{Salanova2011taxi+review}
that have extended it to other taxi markets and factored in more environmental
influences. \textcite{Devany1975taxi+capacity} researched regulated taxi
markets organised as a franchised monopoly, using a medallion system, and
having free entry. With the goal of finding equilibrium output, capacity and
utilisation he suggested a formula tor passenger demand depending on taxi fare,
passenger value of time and waiting time. \textcite{Manski1967taxi+demand}
analysed the taxi market from a purely economical point of view and conclude
that in addition to exogenous variables, passenger demand for taxi services is
also directly related to taxi supply through waiting time. Similarly, taxi
supply is influenced by taxi utilisation, which in turn directly depends on
passenger demand.

\paragraph{}The most recent publications are sophisticated models based on the
network model for cruising-taxi market by \textcite{Yang1998taxi+network}. This
network was modelled as a graph and assumed constant taxi demand and supply,
passenger demand was represented as origin-destination matrices. Finally, this
paper suggested an algorithm to find an equilibrum for the optimal number of
taxis in a market and equations to calculate taxi utilisation and customer
waiting time. In contrast, \textcite{Yang2000taxi+utilization} focused on
supply and demand to recommend optimal policies for taxi regulation in Hong
Kong and based their model on various data sources. A number of exogenous and
endogenous variables affecting taxi market were identified, and equations were
suggested to calculate them: passenger waiting time, percentage of occupied
taxis, vacant taxi headway, daily taxi passenger trips and taxi waiting time.
This model can be used to forecast taxi demand, taxi utilization and service
quality, although the authors warned that it does not take in account all of
the complex supply-demand relationships in taxi market.

\paragraph{}Consequently \textcite{Yang2002taxi+demand} continued to evaluate
the supply-demand equilibria of taxi market started by
\textcite{Yang1998taxi+network} and \textcite{Yang2000taxi+utilization},
resulting in the conclusion that the spatial characteristics of a network where
taxis are operating strongly influence supply and demand, and should bear
weight when evaluating regulatory policies.This study focused on social surplus
(the sum of customer surplus and producer surplus) as the key objective of taxi
markets. Four different regulatory frameworks that could be applied to taxi
markets were investigated: free entry and unconstrained fare, free entry and
regulated fare, regulated entry and unconstrained fare, and regulated entry and
regulated fare. All of these cases were investigated with both competitive and
monopilistic markets, and equilibria were found.

\paragraph{}\textcite{Wong2008taxi+modeling} extended this model to
heterogenous vehicle and user classes, and included congestion which is a major
issue in reality but was ignored by earlier research.
\textcite{Yang2010taxi+nonlinear} proposed a nonlinear fare structure to
correct market and regulatory inefficiencies, and applied it to a similar
model. The way how cruising taxis and customers find each other was researched
by \textcite{Yang2010taxi+equilibria}, paying particular attention to customer
behaviour: this study permitted customers to use other modes of transport e.g.
public transit or walking to find taxis and/or reach their destinations.

\subsubsection{Demand and supply}

\paragraph{} \textcite{Yang2002taxi+demand} cites
\textcite{Manski1967taxi+demand} on the complex structure of demand in taxi
markets, shown in Figure~\ref{figure:taxi}. Both taxi demand and supply are
infulenced by exogenous variables (and regualtion policies, if any). Taxi
demand influences taxi availability and vice versa. Similarly, taxi supply
influences taxi utilization and vice versa. Taxi demand influences taxi
utilization and thus indirectly influences supply, similarly taxi supply
influences taxi availability and thus indirectly influences demand. 

\paragraph{}Customer demand is modelled as a function of waiting time and fare
price in many studies: \textcite{Douglas1972taxi+regulation,
Devany1975taxi+capacity, Cairns1996taxi+competition, Yang2002taxi+demand} use
customer waiting time as a proxy for service quality. According to
\textcite{Salanova2011taxi+review}, \textcite{Manski1967taxi+demand} used a
Poisson process (a probabilistic stohastic function) to simulate demand.
\textcite{Yang2002taxi+demand} model the demand separately for each 
origin-destination pair, where waiting time depends on the number of vacant 
taxis in an area near the customer and price depends on the distance covered;
\textcite{Yang2010taxi+nonlinear} added travel time as an additional variable
indicating service quality and assumed that demand decreases as waiting time
increases. \textcite{Yang2010taxi+equilibria} took a slightly different
approach by modelling customer demand as their willingness to pay to reach a
destination, based on their subjective monetary value for using different modes
of transport for reaching a destination; therefore the demand for taxis in this
study was only a part of the total demand for transportation.


\begin{figure}
  \begin{center}
    \includegraphics{../figures/taxi_demand}
    \caption{
      The demand–-availibility–-utilization–-supply relation in a taxi market. 
      \textbf{NEED PERMISSION}
      \label{figure:taxi}
    }
  \end{center}
\end{figure}
