\subsection{Related works}
\label{sec:literature:related}

\subsubsection{Studies}

\parencite{Emele2013pricing+rural}

\parencite{Kaddoura2013social+cost}

\parencite{Neumann2011berlin}


\subsubsection{Simulation Software}

Hiistorically the most important simulation software has been TRANSIMS
\parencite{Smith1995taxi+transims} and EMME \parencite{Gao2010taxi+comparison}.
TRANSIMS operates on aggregated data in a top-down fashion, and is limited to
certain software modules and specific inputs operating, and does not support
agents. TRANSIMS proved to be useful at its time, but agent-based
microsimulation software has been more popular recently because of the richer functionality. \parencite{Bernhardt2007taxi+agent}

EMME in its latest version 4 is a commercial software, and potentially the most
advanced according to the claims of its parent company. However, it is of
limited interest for this project as EMME is a commercial and closed software.
\parencite{Inro2014emme}

\textcite{Bernhardt2007taxi+agent} gave an overview of agent-based software for
traffic modeling, naming MATSim as the most useful of these. MATSim is the most
noteworthy agent-based simulation and in active development. It has very
extensive capabilities and modules. However, it is complex to use and extend.
\textcite{2010taxi+comparison} claimed that MATSim and EMME/2 has similar
performance and even produce compatible outputs.

ITSUMO by \textcite{Silva2006itsumo} is more limited simulation software using
intelligent agents to model traffic. \textcite{Bazzan2009integrating}
integrated  MATSim and ITSUMO to produce rich large scale traffic simulations.
MATSim, ITSUMO and the integrated piece of software are all meant to be
extendible. Unfortunately none of them have in-built support for agent
interactions that involve prices.


\subsubsection{Traffic Modelling}

\textcite[78]{Bernhardt2007taxi+agent} cited Bonabeau(2002a) for the following
conditions when agent based modelling is appropriate:

\begin{quote}
\begin{itemize}
  \item When there is potential for emergent phenomena--that is, individual 
  behavior is nonlinear and changes over time, and fluctuations are more 
  important than averages;
  \item When describing the system from the perspective of its constituent 
  units’ activities is more natural--that is, “activities are a more natural way
  of describing the system than processes”; and
  \item When the appropriate level of description or complexity is not known 
  ahead of time.
\end{itemize}
\end{quote}

Finally, if an agent-based approach has been deemed appropriate, modelling can
be started. To create models the following needs to be carefully defined: the
environment; agents with their types, attributes, allowable and initial values
for the attributes; agent interaction rules with each other and the
environment. \parencite{Bernhardt2007taxi+agent}
