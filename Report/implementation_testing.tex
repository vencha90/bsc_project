\subsection{Testing}
\label{sec:implementation:testing}

(10, 2100w)

How thorough was the software testing as judged from the description in the
dissertation (e.g. was the software tested for exceptions, scalability, etc.)?


\subsubsection{Automated Tests}
\label{sec:implementation:testing:automated}

Test-Driven Development (TDD) was introduced in Section \ref{sec:design:tdd}.
The TDD process was not followed to the letter. Firstly, in some cases with
unfamiliar and unknown problems tests and code were written intermittently as
uncertainty was slowly reduced. Another reason for not following TDD was the
lax requirements for user interface, where software development alternated
between tests-first and code-first trying to find the easiest solutions to
requirements.

Most of the tests used are black-box unit tests. These tests allow internal
refactoring of software without (theoretically) affecting the overall system
functionality. As expected, black box testing was not sufficient for testing
the Q-Learner where white-box testing was utilized. Additionally, the Taxi
class was white-box tested to an extent, potentially suggesting that it could
have been refactored in separate modules. Integration tests were used for each
top-level class, for example, integration tests for Taxi class tested how it
integrates Action and State classes.

At the time of submission this project's automated test coverage according to
Simplecov was 99.72\% (1053/1056 lines of code covered). Of course, despite the
RSpec tests and Simplecov test coverage, they in no way guarantee correct and
effective code and could even lead to a false sense of security. Nevertheless,
it is very likely that usage of Simplecov has improved overall code quality as
it did identify some code branches that had not been covered by automated
tests.


\subsubsection{User Testing}

user testing was done after software development
was completed. \todo{User testing needs to be done}

complex setup for users not having ruby and on non-unix OS

\subsubsection{Performance Testing}

Only one performance issue was noticed during the usage of the system: A-star
search was chosen for calculating distances between locations. However, this
required repeated graph traversal in the naive form that it was implemented.
This oversight was quickly mitigated by storing the calculated distances in
memory and only traversing the graph for unknown distances.

The main expected software limitation is caused by the dimensionality of state-
action space for Q-Learner. After a certain threshold (system-dependent) it is
likely to exhaust available random access memory and require writing to hard
drive, thus significantly reducing simulation speed. State-action space depends
on the size of the road network and the range of prices available to the taxi.
To be exact, the dimensionality equals \( (n-1) \cdot n \cdot p\), where \(n\)
is number of locations, \(n-1\) is the number of reachable locations and \(p\)
is the number of available prices.

A final less serious issue some users might run into is the considerable
size of simulation log files that have a data row for each time unit of the
simulation. Dealing with this issue is covered in User Manual in Appendix \ref{sec:user_manual}.
