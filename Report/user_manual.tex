\newpage
\section{User Manual}
\label{sec:user_manual}

\todo{remove notes}

(5, 500-1000 w?)

Is the manual written in a direct and easy to understand language?

Does the manual use task-oriented descriptions (e.g. “to open a file you click
on …”)?

Are there walk-through descriptions using detailed examples (e.g. “to enter
customer Joe Bloggs with Address …”)

What is the overall structure, layout, and use of pictures (where appropriate)?

Look at the 'projects' book for advice

\todo{images?}

Taxi-sim is a piece software for evaluation of variable fare pricing in
unregulated taxi markets. It simulates the performance of taxi agents in some
environment and establishes a fixed fare taxi pricing benchmark for reference.
Usage of this software requires basic familiarity with command line terminal.


\subsection{Requirements}

The software is recommended to run on a computer with at least 1 GB of RAM. All
of the installation steps are compulsory unless stated otherwise. A fast
internet connection is required for installation, and as it relies on external
services a 100\% availability cannot be guaranteed.


\subsection{Installation}

Commands that should be used in command line terminal are marked
\textit{(cmd)}, commands that require downloading software from the world wide
web are marked \textit{(web)}.

\begin{itemize}
  \item \textit{(web)} Install Git \parencite{Git} version control from 
        http://git-scm.com/.
  \item \textit{(cmd)} Open a command line terminal.
  \item \textit{(cmd)} Clone source code to a suitable directory on your local
        hard drive using this command: 
        \textit{git clone git@github.com:vencha90/taxi-sim.git}
  \item \textit{(cmd)} Change the working directory to the project's source
        code: \textit{cd taxi-sim}. Taxi-sim is the name of the software.
\end{itemize}

It is recommended that you use virtualisation as described in Section
\ref{sec:user_manual:installation:virtualisation}. This will ensure that only
the correct required dependencies are installed and your operating system is
not corrupted by potentially conflicting software. 

If you still do not wish to use virtualisation, there are alternatives for
unix-like systems. You can install RVM \parencite{Rvm} from http://rvm.io/ and
then install Ruby by running \textit{rvm install 2.1.1} in terminal. You also
need to install Bundler \parencite{Bundler} by running \textit{gem install
bundler && bundle install} to manage dependencies. Instead of RVM you could use
\textit{rbenv} \parencite{Rbenv}.


\subsubsection{Virtualisation}
\label{sec:user_manual:installation:virtualisation}

Virtualisation is recommended for most unix-like operating systems (including
Mac OS X and Linux derivatives), but absolutely required for Windows. At the
time of writing, Windows does not support Ruby 2.1.1 which is required for this
software. Even when 2.1 support is added, using Windows natively could cause
you nasty surprises!

Vagrant's configuration used here in \textit{Vagrantfile} and
\textit{vagrant_manifest.pp} is adapted from the work of
\textcite{Rails+dev+box}.

\begin{itemize}
  \item \textit{(web)} Install VirtualBox version 4.3+
        \parencite{Virtualbox} from
        https://www.virtualbox.org/ using the instructions on the website.
  \item \textit{(web)} Install Vagrant version 1.5+ \parencite{Vagrant} from 
        http://www.vagrantup.com/ following instructions on the website.
  \item \textit{(cmd)} In the taxi-sim directory, run \textit{vagrant up}. This
        will provision and start the virtual machine. It will take slightly
        longer the first time as the operating system image needs to be
        downloaded.
  \item \textit{(cmd)} \textit{vagrant ssh} will connect to the virtual machine 
        using a secure shell connection, giving you control of the virtual
        machine's terminal.
  \item \textit{(cmd)} \textit{cd /vagrant/} will change the working
        directory to the software. Vagrant synchronises this directory with the
        software directory you cloned to your machine a few steps ago.
  \item \textit{(cmd)} The final step before you can use Taxi-sim is installing
        the latest \textit{gem} (software library) dependencies by running 
        \textit{bundle install}. When this step is finished you are ready to
        use Taxi-sim!
\end{itemize}


\subsection{Using Taxi-sim}
\label{sec:user_manual:using}

Taxi-sim needs to be used from command line. If you are using Vagrant as
recommended in Section \ref{sec:user_manual:installation:virtualisation}, run
\textit{vagrant up && vagrant ssh && cd /vagrant/ && bundle install && bundle
exec rspec} to check that everything is in order. You might notice that
\textit{bundle exec} is used at the start of most commands: this ensures that
the commands are executed by bundler using the correct gem (software library)
versions as a system could have multiple gem versions insatlled.


\subsubsection{Interface}
\label{sec:user_manual:using:interface}

Taxi-sim supports a simple command line interface. You can see the supported
tasks and a short description by running \textit{bundle exec rake -T}.

\textit{bundle exec rake spec} runs the automated test suite and is useful for
verifying the integrity of the software, especially if you have made any
changes.

\textit{bundle exec rake run[path/to/file]} is the main command for you as a
user. You need to specify a path to an input file in the square brackets
relative to the working directory, e.g. \textit{rake run[sample_input.yml]}.
The accepted file format is YML. Open \textit{sample_input.yml} in a text
editor to see a sample input file, a detailed explanation of the accepted
inputs is in Section \ref{sec:user_manual:using:inputs}


\subsubsection{Inputs}
\label{sec:user_manual:using:inputs}

All types of inputs supported by the simulation are used in
\textit{sample_input.yml} file, their order does not matter. They are explained
in the following list in alphabetical order:

\begin{itemize}
  \item \textit{graph}: an adjacency matrix representation of a road network.
        It has to be a connected undirected graph, meaning that the matrix is
        symmetrical! Self-loops are ignored. The accepted format for each row
        is \textit{- "x, .., z"} where \textit{x} and \textit{z} are integer
        number edge weights.
  \item \textit{passenger} is a top-level entity for passenger details and has
        this parameter:
        \begin{itemize} 
          \item \textit{price} is an integer number of the price that
                passengers expect to pay for a single unit of distance, used to
                calculate probability of accepting a fare as described in
                Section \ref{sec:requirements:passenger}.
        \end{itemize}
  \item \textit{taxi} is a top-level entity for taxi details and has the
        following parameters:
        \begin{itemize}
          \item \textit{benchmark_price}: an integer number of the fixed price
          that the simulation will be benchmarked against as described in
          Section \ref{sec:requirements:benchmark}.
          \item \textit{prices}: a range of prices that the taxi can choose
          from. An integer number, an array of integer numbers in the format 
          \textit{[x, .., z]}, or a range of integer numbers in the Ruby range
          format \textit{"x..z"} (includes \textit{z}) or \textit{"x...z"}
          (excludes \textit{z}).
        \end{itemize}
  \item \textit{time__limit}: an integer number for the units of time the
        simulation will run for
\end{itemize}


\subsubsection{Output files}

\todo{large output file} Simulation output is written to \textit{logs}
directory in the application's home directory. These files can be opened using
most text editing software, although some mainstream software might crash due
to the large file size of \textit{simulation.log}. In this case software that
supports huge file sizes can be easily found, for example, notepad++
\parencite{Notepad++}.

\textit{summary.log} is a summary of the simulation results with six rows. The
first three rows state the time limit, taxi details and the total profit for
the taxi using a range of prices. The last three rows state the same data for
the benchmark run with a fixed price.

\textit{simulation.log} is a full log of the simulation as it progressed. It
has two parts, the first being the run with a range of prices and the second
run with a fixed taxi price. It has a single row for each time unit in both
runs, therefore this file is likely to only be useful for data analysis.

If you require customised output, the data written to log files can be changed
by changing just a few lines of the code. This is explained in the Maintenance
Manual in Appendix \ref{sec:maintenance_manual:customising_output}.
