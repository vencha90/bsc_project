\newpage
\section{User Manual}
\label{sec:user_manual}

\todo{remove notes}

(5, 500-1000 w?)

Is the manual written in a direct and easy to understand language?

Does the manual use task-oriented descriptions (e.g. “to open a file you click
on …”)?

Are there walk-through descriptions using detailed examples (e.g. “to enter
customer Joe Bloggs with Address …”)

What is the overall structure, layout, and use of pictures (where appropriate)?

Look at the 'projects' book for advice

\subsection{Requirements}

The software is recommended to run on a computer with at least 1 GB of RAM. All
of the installation steps are compulsory unless stated otherwise. A fast
internet connection is required for installation, and it relies on external
services so a 100\% availability cannot be guaranteed.


\subsection{Installation}

Commands that should be used in command line terminal are marked
\textit{(cmd)}, commands that require downloading software from the world wide
web are marked \textit{(web)}.

\begin{itemize}
  \item \textit{(web)} Install Git \parencite{Git} version control from 
        http://git-scm.com/.
  \item \textit{(cmd)} Open a command line terminal.
  \item \textit{(cmd)} Clone source code to a suitable directory on your local
        hard drive using this command: 
        \textit{git clone git@github.com:vencha90/taxi-sim.git}
  \item \textit{(cmd)} Change the working directory to the project's source
        code: \textit{cd taxi-sim}. Taxi-sim is the name of the software.
\end{itemize}


\subsubsection{Virtualisation}
\label{sec:user_manual:installation:virtualisation}

Virtualisation is recommended for most unix-like operating systems (including
Mac OS X and Linux derivatives), but absolutely required for Windows. At the
time of writing, Windows does not support Ruby 2.1.1 which is required for this
software. Even when 2.1 support is added, using Windows natively could cause
you nasty surprises! If you have a unix-like system and do not want to use
virtualisation, skip the list to the last paragraph.

Vagrant's configuration used here in \textit{Vagrantfile} and
\textit{vagrant_manifest.pp} is adapted from the work of
\textcite{Rails+dev+box}.

\begin{itemize}
  \item \textit{(web)} Install VirtualBox version 4.3+
        \parencite{Virtualbox} from
        https://www.virtualbox.org/ using the instructions on the website.
  \item \textit{(web)} Install Vagrant version 1.5+ \parencite{Vagrant} from 
        http://www.vagrantup.com/ following instructions on the website.
  \item \textit{(cmd)} In the taxi-sim directory, run \textit{vagrant up}. This
        will provision and start the virtual machine. It will take slightly
        longer the first time as the operating system image needs to be
        downloaded.
  \item \textit{(cmd)} \textit{vagrant ssh} will connect to the virtual machine 
        using a secure shell connection, giving you control of the virtual
        machine's terminal.
  \item \textit{(cmd)} \textit{cd /vagrant/} will change the working
        directory to the software. Vagrant synchronises this directory with the
        software directory you cloned to your machine a few steps ago.
  \item \textit{(cmd)} The final step before you can use Taxi-sim is installing
        the latest dependencies by running \textit{bundle install}. When this
        step is finished you are ready to use Taxi-sim!
\end{itemize}

Not recommended: alternative for unix-like systems. You can install RVM
\parencite{Rvm} from http://rvm.io/ and then install Ruby by running
\textit{rvm install 2.1.1} in terminal. You also need to install Bundler
\parencite{Bundler} by running \textit{gem install bundler && bundle install}
to manage dependencies.


\subsection{Using Taxi-sim}

Taxi-sim needs to be used from command line. If you are using Vagrant as
recommended in Section \ref{sec:user_manual:installation:virtualisation}, run
\textit{vagrant up && vagrant ssh && cd /vagrant/ && bundle install &&
bundle exec rspec} to check that everything is in order.

\subsubsection{Interface}

Taxi-sim supports a simple command line interface. You can see the supported
tasks by running \textit{bundle exec rake -T}.

\textit{rake spec} runs the automated test suite and is useful for verifying
the integrity of the software, especially if you have made any changes.

\textit{rake run[path/to/file]} is the main command for you as a user. You need
to specify a path to file in the square brackets relative to the working
directory, e.g. textit{rake run[some_file.yml]}. The accepted file format is
YML. To see a sample input file, open \textit{sample_input.yml} in a text
editor.

\todo{more specific references to user manual from the report}

\todo{input file}

\todo{large output file}
Some mainstream software might crash due to large file sizes, but compatible 
software can be easily found such as notepad++ \todo{reference}

\todo{output analysis}
