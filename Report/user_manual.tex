\newpage
\section{User Manual}
\label{sec:user_manual}

(5, 500-1000 w?)

Is the manual written in a direct and easy to understand language?

Does the manual use task-oriented descriptions (e.g. “to open a file you click
on …”)?

Are there walk-through descriptions using detailed examples (e.g. “to enter
customer Joe Bloggs with Address …”)

What is the overall structure, layout, and use of pictures (where appropriate)?

Look at the 'projects' book for advice

\todo{remove notes}

Taxi-sim is a piece software for evaluation of variable fare pricing in
unregulated taxi markets. It simulates the performance of taxi agents in some
environment and establishes a fixed fare taxi pricing benchmark for reference.
Usage of this software requires basic familiarity with command line terminal.

This User Manual starts with installation instructions in Section
\ref{sec:user_manual:installation}. Instructions for operating the software are
in Section \ref{sec:user_manual:using}.


\subsection{Installation}
\label{sec:user_manual:installation}

If you do not have the source code of the software available, it can be
downloaded as a \textit{zip} archive from \url{https://github.com/vencha90
/taxi-sim/archive/master.zip}.

For running the software it is recommended that you use virtualisation as
described in Section \ref{sec:user_manual:installation:virtualisation}. If you
wish to install the software stack to your operating system natively, please
see Maintenance Manual in Appendix \ref{sec:maintenance_manual:native_install}.


\subsubsection{Requirements}

The software is recommended to run on a computer with at least 1 GB of RAM. All
of the installation steps are compulsory unless stated otherwise. A fast
internet connection could be required for installation of some supporting
software, and as it relies on external services a 100\% availability cannot be
guaranteed.

Operating systems that are supported are: Mac OS X, Windows, Debian / Ubuntu,
CentOs / RedHat / Fedora.


\subsubsection{Virtualisation}
\label{sec:user_manual:installation:virtualisation}

Virtualisation is recommended for most Unix-like operating systems (including
Mac OS X and Linux derivatives), but absolutely required for Windows. At the
time of writing, Windows does not support Ruby 2.1.1 which is required for this
software.

Vagrant's configuration files \texttt{Vagrantfile} and
\texttt{vagrant\_manifest.pp} used here are adapted from the work of
\textcite{Rails+dev+box}.

\begin{itemize}
  \item \textit{(web)} Install VirtualBox version 4.3+ \parencite{Virtualbox}
        from \url{https://www.virtualbox.org/} using the instructions on the
        website.
  \item \textit{(web)} Install Vagrant version 1.5+ \parencite{Vagrant} from 
        \url{http://www.vagrantup.com/} following instructions on the website.
  \item \textit{(cmd)} In the taxi-sim directory, run \texttt{vagrant up}. This
        will provision and start the virtual machine. It will take slightly
        longer the first time as the operating system image needs to be
        downloaded.
  \item \textit{(cmd)} \texttt{vagrant ssh} will connect to the virtual machine 
        using a secure shell connection, giving you control of the virtual
        machine's command line terminal.
  \item \textit{(cmd)} \texttt{cd /vagrant/} will change the working
        directory to the software. Vagrant synchronises this directory with the
        software directory you cloned to your machine a few steps ago.
  \item \textit{(cmd)} The final step before you can use Taxi-sim is installing
        the \textit{gem} (software library) dependencies by running 
        \texttt{bundle install --local}. When this step is finished you are
        ready to use Taxi-sim!
\end{itemize}

If you can successfully \texttt{ssh} on the virtual machine but further steps
are not working, please try the following steps:
\begin{itemize}
  \item log out of the virtual machine by entering \texttt{exit},
  \item delete the virtual machine by entering \texttt{vagrant destroy},
  \item recreate the virtual machine by entering \texttt{vagrant up},
  \item continue as normally.
\end{itemize}


\subsection{Using Taxi-sim}
\label{sec:user_manual:using}

Taxi-sim needs to be used from command line. If you are using Vagrant as
recommended in Section \ref{sec:user_manual:installation:virtualisation}, in
terminal from the main software directory run \texttt{vagrant up \&\& vagrant
ssh \&\& cd /vagrant/ \&\& bundle install --local \&\& bundle exec rspec} to
check that everything is in order.

You might notice that \texttt{bundle exec} is used at the start of most
commands: this ensures that the commands are executed by bundler (dependency
manager software) using the correct gem (software library) versions as a system
could have multiple gem versions installed.

This section covers usage of Taxi-sim. To find out how to operate the software
from command line terminal, see Section \ref{sec:user_manual:using:interface}.
For configuring the simulation inputs see Section
\ref{sec:user_manual:using:inputs}. You can find out about the structure of
output files in Section \ref{sec:user_manual:using:output}. An easy way to
analyse the output data is suggested in Section
\ref{sec:user_manual:using:data}.

\subsubsection{Interface}
\label{sec:user_manual:using:interface}

Taxi-sim supports a simple command line interface. You can see the supported
tasks and a short description by running \texttt{bundle exec rake -T}.

\texttt{bundle exec rake spec} runs the automated test suite and is useful for
verifying the integrity of the software, especially if you have made any
changes.

\texttt{bundle exec rake run[path/to/file]} is the main command for you as a
user. You need to specify a path to an input file in the square brackets
relative to the working directory, e.g. \texttt{rake run[sample\_input.yml]}.
The accepted file format is YML. Open \texttt{sample\_input.yml} in a text
editor to see a sample input file, a detailed explanation of the accepted
inputs is in Section \ref{sec:user_manual:using:inputs}.


\subsubsection{Inputs}
\label{sec:user_manual:using:inputs}

All types of inputs supported by the simulation are used in
\texttt{sample\_input.yml} file, their order does not matter. They are explained
in the following list in alphabetical order:

\begin{itemize}
  \item \textit{graph}: an adjacency matrix representation of a road network.
        It has to be a connected undirected graph, meaning that the matrix is
        symmetrical! Self-loops are ignored. The accepted format for each row
        is \textit{- "x, .., z"} where \textit{x} and \textit{z} are integer
        number edge weights.
  \item \textit{passenger} is a top-level entity for passenger details and has
        this parameter:
        \begin{itemize} 
          \item \textit{price} is an integer number of the price that
                passengers expect to pay for a single unit of distance, used to
                calculate probability of accepting a fare as described in
                Section \ref{sec:requirements:passenger}.
        \end{itemize}
  \item \textit{taxi} is a top-level entity for taxi details and has the
        following parameters:
        \begin{itemize}
          \item \textit{benchmark\_price}: an integer number of the fixed price
          that the simulation will be benchmarked against as described in
          Section \ref{sec:requirements:benchmark}.
          \item \textit{prices}: a range of prices that the taxi can choose
          from. An integer number, an array of integer numbers in the format 
          \textit{[x, .., z]}, or a range of integer numbers in the Ruby range
          format \textit{"x..z"} (includes \textit{z}) or \textit{"x...z"}
          (excludes \textit{z}).
        \end{itemize}
  \item \textit{time\_limit}: an integer number for the units of time the
        simulation will run for
\end{itemize}

A sample input file is shown in Figure \ref{figure:input}.

\begin{figure}
  \begin{verbatim}
    time_limit: 1000000
    graph:
      - "0, 1, 2, 4, 2, 1"
      - "1, 0, 2, 2, 3, 0"
      - "2, 2, 0, 0, 1, 3"
      - "4, 2, 0, 0, 1, 0"
      - "2, 3, 1, 1, 0, 5"
      - "1, 0, 3, 0, 5, 0"
    passenger:
      price: 20
    taxi:
      prices: '5..100'
      benchmark_price: 20
  \end{verbatim}
\caption{
  Sample Input File
  \label{figure:input}
}
\end{figure}


\subsubsection{Output files}
\label{sec:user_manual:using:output}

Simulation output is written to \texttt{logs/} directory in the application's
home directory. These files can be opened using most text editing software,
although some mainstream software might crash due to the large file size of
\texttt{simulation.log}. In this case software that supports huge file sizes
can be easily found, for example, notepad++ \parencite{Notepad++}. Parameters
are logged in this format: \texttt{["name", "value"]}.

\texttt{summary.log} is a summary of the simulation results with six rows. The
first row is written when the simulation starts stating the time limit, the
next two rows state taxi details and the total profit for the taxi using a
range of prices. The last two rows state the same data for the benchmark run
with a fixed price. Contents of a sample summary file are shown in Figure
\ref{figure:output:summary}.

\begin{figure}
  \begin{verbatim}
["time", "0"]["time_limit", "1000000"]
["fc", "1"]["vc", "1"]["prices", "5..100"]
["time", "1000000"]["profit", "839289"]
["fc", "1"]["vc", "1"]["prices", "[20]"]
["time", "1000000"]["profit", "-1677056"]
  \end{verbatim}
\caption{
  Sample Simulation Summary File
  \label{figure:output:summary}
}
\end{figure}

\texttt{simulation.log} is a full log of the simulation as it progressed. It
has two parts, the first being the run with a range of prices and the second
run with a fixed taxi price. It has a single row for each time unit in both
runs, therefore this file is likely to only be useful for data analysis. An
excerpt from a simulation log is shown in Figure \ref{figure:output:log}.

\begin{figure}
  \begin{verbatim}
["time", "155"]["reward", "-4"]["location", "5"]["busy_for", "3"]["action", "drive"]
["time", "156"]
["time", "157"]
["time", "158"]["passenger", "accepted"]["fare", "15"]["location", "5"]
        ["destination", "6"]["reward", "9"]["location", "6"]
        ["busy_for", "3"]["action", "offer"]
  \end{verbatim}
\caption{
  Excerpt from a Sample Simulation Log File
  \label{figure:output:log}
}
\end{figure}

If you require customised output, the data written to log files can be changed
by modifying just a few lines of the code. This is explained in the Maintenance
Manual in Appendix \ref{sec:maintenance_manual:customising_output}.


\subsubsection{Data Analysis}
\label{sec:user_manual:using:data}

\todo{explain usage} 

The R Project provides an open-source statistical programming software package
\parencite{Rlang}. An R language script file \texttt{analysis.R} for Taxi-sim
data analysis is included in the project source code.

The first step before the script can be used is separation of benchmark data
and variable pricing data from the main data source. Open
\texttt{simulation.log} in a text editor and make two new files by splitting
the original file in half by copy-pasting each run's data. Name the file with
variable prices \texttt{variable.log} and the file with fixed price
\texttt{benchmark.log}, and store them in the same directory. For the script to
work, you have to delete the first and last line of \texttt{variable.log} where
no actions are selected.

Open the script file in an R work environment. If you followed the step above,
you only have to modify line 4. Change the path of your working directory in
\texttt{setwd("your/path")}, for example, to be
\texttt{setwd("c:\\User/Documents/Taxi-sim/Data/")} \todo{change the file to
match}

Run the script -- it will take some time depending on the amount of data. This
is a memory expensive operation and depends on the available RAM. When the
script is finished, it will print a summary pdf file with graphs and
descriptive statistics. This script uses external libraries available online,
therefore an internet connection is required. It is expected that you will see
warnings about removal of empty rows -- the script measures reward that was
empty when taxi had been busy with some actions.
