\newpage
\section{Requirements}
\label{sec:requirements}

\todo{remove note}
(5, 1000w)

Has the student appropriately analysed and defined the requirements for the
project and presented them clearly?


\subsection{Functional Requirements}
\label{sec:requirements:simulation}

The following requirements are derived directly from the project's goals set in
Section \ref{sec:intro:goals}. Alas, it is unfeasible to structure this section
around each goal as they contain several requrements that sometimes overlap
with other goals. Each of the following requirement categories is discussed in
detail, listing what could be implemented in the best case scenario without any
time pressure. However, to stay realistic and make sure that core functional
requirements are prioritised, the minimal requirements are separately summed up
in Table \ref{table:requirements} for each of the following sections.


\subsubsection{Market Considerations}

The approach suggested by this project is not compatible with a market where
fares are regulated, at least in the current form of regulation that specifies
a formula to calculate fares based on some variables, usually time and
distance. Other ways of regulation that do not affect pricing, for example,
market entry conditions, are compatible with the suggested variable pricing
approach.

Three different operational types of taxi markets were introduced in Section
\ref{sec:literature:taxis}: phone-order market, cruising taxi market and taxi
rank market. In the phone-order market and taxi rank customers are actively
seeking a taxi, while in the cruising taxi market passengers can only wait for
a taxi to drive by. Therefore the cruising taxi market is chosen as the easiest
target for simulation, extending it to phone-order and taxi rank markets if the
initial experiment is successful.

The relationship between taxi demand and supply and vice versa was established
in Section \ref{sec:literature:taxis:demand}. This relationship largely depends
on the competitive situation in a taxi market. This is a very complex
relationship and thus implementing it in software would require considerable
effort. Because the core goal of the project is comparing variable pricing
approaches, not correctly modelling the market, this problem can be avoided by
assuming a constant demand. Furthermore, if the environment is fixed, the
simulation only needs to support a single taxi at the very least. As the design
is planned modular, the variable demand can be factored in later when
necessary.


\subsubsection{Demand: Modelling Passengers}
\label{sec:requirements:passenger}

Customer demand was reviewed in Section \ref{sec:literature:taxis:demand} where
two approaches to modelling demand were shown: aggregate and disaggregate. An
aggregate demand model "for some portion of the travel market is a function of
variables that describe the product or its consumers"
\parencite{Small2007taxi+urban} The disaggregate approach specifies a set of
variables for each individual passenger. The project goal is investigating taxi
pricing on an individual basis, therefore an agent-based approach is
preferrable as it allows individual modelling of passengers. Of course, to
express demand as a single variable it needs to be an aggregation of the
relevant individual variables.

Therefore each passenger's demand is a function that of some variables. The
different types of variables affecting taxi demand were discussed in Section
\ref{sec:literature:taxis:demand}. Exogenous demand variables are value of time
and value of reliability that can be derived from a passenger's income, hour of
day when travelling, purpose of travel, social status, cost of waiting and
others; these can be modelled for each individual passenger. Taxi availability
is a variable directly depending on the number of vacant taxis in an area, this
is something that passenger's perceive in reality. However, for the simulation,
taxi availability needs to be assumed a constant, at least until machine
learning capabilities are added to passengers and a competitive market
established beyond the very basic simulation, so that passanger agents can
learn the availability on their own.

Let \(P\) be the set of relevant characteristic variables \(c_1,..,c_n\) for a
passenger: \(P = \{c_1,..,c_n\}\), \newline
where each \(c_i : i \in \{1, .., n\}\) has a function \(f_i (c_i) \) that
returns a unity-based normalised value of \(c_i\), \newline
and each \(c_i\) has a weight \(w_i\) representing its relative importance
compared to other variables, where \(\sum_{i=1}^n (w_i) = 1 \). \newline
Then the total probablilstic value \(V(P)\) of a passenger is \[ V(P) =
\sum_{i=1}^{n} (w_i \cdot f_i(c_i) )) \] \newline

Let \(\Delta(o,d)\) be the distance between origin \(o\) to destination \(d\). 
\newline
Let the passenger's expected fare \(F_{expect}\) linearly depend on distance
and a set price \(p\) for a unit of distance. \(F_{expect} = \Delta(o,d) \cdot
p \). \newline
Let the price sensitivity of customers be \(\frac{F_{expect}}{F_{offer}}\). It
is worth noting that a price sensitivity value specific to an individual could
be already included in the passenger characteristics and this is just a global
representation for the whole of system. \newline

Then the probability of a customer accepting a fare \(Q\) for a taxi ride at a
fare price \(F_{offer}\) can be expressed as: 
\begin{equation} 
  \label{eq:requirements:demand}
  Q_{o \rightarrow d} (P,F_{offer}) = \frac{F_{expect}}{F_{offer}} \cdot V(P)
\end{equation}
This formula is sufficient for the basic case of simulation. It can later be
extended to take in account other determinants of demand as seen in Figure
\ref{figure:taxi} and discussed in Section \ref{sec:literature:taxis:demand}.

The relevant variables for passengers can be generated using a stochastic
process. Similarly, passenger distribution within the network can be generated
using a stochastic process. These processes could take in account some
characteristics observed in reality such as demand variance during the day,
lower passenger income in some areas resulting in lower willingness to pay,
whether a trip is for pleasure or business, and others.


\subsubsection{Taxis in Taxi Market}
\label{sec:requirements:taxi}

Let taxis have some variable costs \(VC\) for a unit of time directly related
to driving and some of fixed costs \(FC\) for a unit of time. Examples of fixed
costs are business overheads, insurance and deprecation. Examples of variable
costs is fuel costs and taxi running costs. In this case drivers' wages are a
fixed cost because it is assumed that the taxi is always active. For simplicity
it is assumed that the variable cost depends only on the distance covered
\(d\). Then total taxi costs \(TC\) for a total time \(t\) can be expressed as:
\[ TC = t(VC\cdot d + FC) \]

Taxis have a set of available actions. When stopped at a location, they can
decide to drive to another location or wait. If there is a passenger present,
they can start interacting with the passenger, ask for a desired destination
and offer a price.

When taxis interact with customers in reality in a market with no fare
regulation, bargaining is likely to happen: passengers state a destination,
taxis bid passengers a fare, and passengers can agree with it, or decline it
and give a countering bid or abandon the process. Bargaining allows taxis and
passengers to agree on a mutually acceptable price. To simulate real-world
behaviour, a reinforcement learning-based bargaining process could be used as
described in \textcite{Cli1997taxi+bargaining}.

However, sophisticated bargaining can be disregarded in the simulation if the
horizon is significantly long as the agreed fares should converge, albeit at a
slower rate. A simpler approach is limiting the bargaining process to a single
bid which is immediately accepted or declined by the passenger based on the
demand \(Q\). To incentivise the taxi agent, each bid has a cost in time.


\subsubsection{Modelling reinforcement learning}
\label{sec:requirements:ai}

The research problem needs to be formally defined from a reinforcement learning
point of view according to the definitions in Sections
\ref{sec:literature:ai:mdp} and \ref{sec:literature:ai:pomdp}. Maximising the
profit can be chosen as the goal of the taxi agent's operation.

It can be assumed that taxi has a complete knowledge of the road network it is
operating in - a realistic assumption given modern GPS navigation systems. In
reality this set would be infinite but it can be simplified to a finite set.
This network forms a part of the total state space. The rest of the state space
is formed of the passenger origin-destination pairs i.e. some state \(s = (o,
d) \). The passenger demand for each of these origin-destination states is
stochastic as mentioned in Section \ref{sec:requirements:passenger}. 

The actions that a taxi can take were discussed in Section
\ref{sec:requirements:taxi}. A simple reward function is as follows. For each
moment of time that a taxi is active, there is an immediate negative reward
based on fixed costs and salary variable costs. If the taxi takes travels, it
suffers a negative reward based on travel variable costs. The only positive
reward it gains is from passengers paying their fares.

If passengers wait at a location for a longer time, then taxis can form
expectations about the possibility of encountering passengers at each location.
To keep track of this uncertainty an agent would need a belief model as
discussed in Section \ref{sec:literature:ai:pomdp}. This complexity can be
avoided by assuming that passengers do not wait for taxis and appear at a
location for a single moment in time only.

As recognised in Section \ref{sec:literature:ai:mdp:methods}, the simplest
reinforcement learning method to implement is Q-learner, and it has been just
proven to be adequate for the minimum requirements when the complexity is
reduced.


\subsubsection{Benchmark} 

To evaluate the variable pricing approach, benchmarks measuring taxi
profitability need to be established using the linear pricing approach. It will
establish the average best-case scenario for both passengers and taxis. If the
reinforcement learning agent can reach or exceed similar profits it will mean
that the experiment was a success.

As noted in Section \ref{sec:literature:taxis:demand}, the demand and supply
relationship is complex in taxi market. Therefore an equilibrum cannot be
easily calculated and is unlikely to have real value. However, the expected
fare from the Equation \ref{eq:requirements:demand} linearly depends on
distance just like taxi costs, therefore it is a good indicator for taxi
efficiency. The exact tarriff can be calculated by fixing probability
distributions for the stochastic process and using average values.

When a tariff is chosen, the simulation can be run the exact same way as with a
reinforcement learning, the only difference being the agent having no say about
the fare. As the agent faces the same costs in both simulations, costs need not
be taken in account.


\begin{table}
\begin{tabular}{| p{0.15 \textwidth} | p{0.8 \textwidth} |}
  \hline
  
  Market considerations & A fixed environment with a fixed demand for taxi
  services, that supports a single taxi cruising in a road network \\ \hline

  Passengers & Passive agents that are stochastically generated at random
  points in the environment. Each passenger has a destination they wish to
  travel to, and a set of user-specified characteristics determining if they
  agree to pay a certain fare for the taxi ride \\ \hline

  Taxi & An active agent that have fixed and variable costs, and that can travel
  in the environment. Taxi has a set of actions that it can take: offer a fare
  to a passenger (if one is present), wait at a location or drive to another
  location. If the offer is accepted, the taxi drives the passenger totheir
  destination, but the passenger is lost if the offer was declined. \\ \hline

  Reinforcement Learning & A Q-learner agent for the taxi that interacts with
  the environment, the taxi and the passengers with a goal to maximise the
  taxi's profit \\ \hline

  Benchmark & The simulation supports a taxi agent with a single fixed price \\
  \hline
\end{tabular}
\caption{
  Summary of Functional Requirements
  \label{table:requirements}
}
\end{table}

\subsection{Non-Functional Requirements}

\textbf{Maintainability:} the system needs to be maintainable and extendable.
In addition to a maintenance manual targeted at developers, the code needs to
be documented. Software needs to be modular, so that components can be easily
exchanged. The development process needs to be transparent so that history and
context can be easily traced if need arises, for example, to fix a bug.

\textbf{Data:} all properties of the simulation need to be logged so that they
can later be analysed. This means logging every value that has changed at each
iteration.
