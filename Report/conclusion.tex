\newpage
\section{Conclusion}
\label{sec:conclusion}

This project was mostly successful in implementing a software simulation of a
taxi market where fare prices are chosen by a Q-Learner agent. 

The software fulfilled all of the core goals to the minimum specification, and
exceeded the minimum requirements in some areas. The software suffers from
several small implementation issues that have been clearly identified, and a
plan for fixing these issues was provided. While some problems were identified
with the software development approach that was followed, it was adequate and
proved beneficial overall.

The simulation experiment was successfuly completed with six different
scenarios, testing different aspects of the system. It was found that having
variable fare pricing results in significantly larger profits to a taxi.
Furthermore, it was clear that the Q-Learner was a reasonable choice for
controlling the taxi and it's fare pricing as it showed a clear trend of
increasing profits over time.

However, the results additionally revealed an inconclusive profitability trend
for the Q-Learner in the fixed pricing simulation scenarios conducted for
benchmarking. This suggests that there may be a bug or a design oversight in
the software which will need fixing in future.

Nevertheless, variable pricing proved to be more profitable than fixed pricing,
therefore opening future possibilities for the project's topic and software.
Possible future work includes fixing the known issues with the software,
experimenting with different reinforcement learning strategies or even
researching the wider social implications of a variable fare pricing approach.
