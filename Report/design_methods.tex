\subsection{Methodology}

\subsubsection{Test-Driven Development}
\label{sec:design:tdd}

The author has previous satisfactory industrial experience of working with the
Agile software development philosophy (Scrum framework) and following the Test-
Driven Development methodology (TDD). Scrum has a lot in common with eXtreme
Programming (XP) methodology \parencite{Copeland2001xp}, particularly its
software development aspects. Agile philosophy, Scrum and XP are most useful
for teams working on larger projects \parencite{Cohn2010agile}, so they will
not be discussed here any more. Still, TDD is one of the practices recommended
by Scrum (or XP) to improve software quality, and apart from most parts of
Scrum (and XP) TDD is aimed at individual software developers
\parencite{Beck2000xp, Cohn2010agile}.

The TDD cycle goes as follows: automated tests are written first, followed by
writing program code which is finally refactored to improve quality -- this
results in a complete test coverage and a regression test suite which actually
enables safe refactoring. As an added benefit the automated test suite serves
as documentation for the code and helps explain developers' reasoning at the
time of writing software. \parencite{Beck2000xp}

Studies have found that TDD is better for code quality than traditional non-TDD
fashion: \textcite{Williams2003test} found that software produced in IBM
following TDD had 40\% less bugs, and \textcite{Bhat2006tdd} found a 50\%
increase in code quality in two different environments at Microsoft.

However, \textcite{Bhat2006tdd} also notes that TDD required 15\% more time
upfront to write tests. Finally, \textcite{Dhh2014tdd} warns about blindly
following TDD and not employing other tools and practices that could aid
software development.

Therefore it was chosen to follow TDD as closely as possible without
compromising other aspects of successful software development. This means that
the traditional Waterfall methodology phases of requirement specification,
design and testing still apply to this project.

TDD is heavily focused on black-box unit testing, but does not exclude other
ways of testing software. White-box unit testing was needed to test complex
parts of the software such as Q-Learner. Furthermore, integration testing is
needed to ensure that the system works as a whole.

The technologies used for TDD are discussed in Section
\ref{sec:design:software:tdd}. The implemented tests are discussed in more
detail in Section \ref{sec:implementation:testing} with actual examples and
discussion of how the author managed TDD.
