\subsection{Software design}
\label{sec:design:software}

Global properties: time; Taxis
can enquire for the route from A to B at a small cost, and receive distance.
When taxis decide to go from A to B, they go at a constant speed for the
calculated time. No actions can be taken and they are marked as travelling.

Network of nodes, not known to taxi as a whole

Node has a list of neighbours and distances.
Node has a list of present passengers and taxis.

Taxi model: fixed and variable costs, has a set of actions:
getDistance(A,B),drive(A,B), askDestination(P), offerFare(P).

Passenger model: probabilistic properties with set relative weights to
calculate demand, responds to bids with Y/N.

\subsubsection{A Road Network}
\label{sec:design:network}

The market is represented as a weighted connected graph constructed of edges
(roads) and vertices (origins/destinations). Each vertex has associated
properties: a list of connected edges with their lengths (weights), a list of
passengers, potentially a list of taxis. Each edge has associated properties:
length, potentially speed limit and toll. Taxis can travel between vertices
using edges. Taxis are assumed to take the shortest routes and travel at a
constant speed. Passengers appear at nodes and when taxi is at a node it is
assumed that they can interact, and the passengers wait for a set amount of
time.

Road networks have a specific structure that is different from generic randomly
generated graphs. \textcite{Eisenstat2011graphs+quadtree} believes that the
structure of the graphs is important for optimisation problems, giving the
example of optimising logistics operations for a fleet of vehicles where
algorithmic performance greatly differs from that on generic graphs. Uniform
planar graphs are a reasonably realistic way to represent road networks
\parencite{Eisenstat2011graphs+quadtree, Masucci2009graphs+london}. A way to
generate random planar graphs is suggested by
\textcite{Brinkmann2007graphs+generate}, for which ready-to-use software called
\textit{plantri} is available. 

Shortest paths in graphs can be calculated by Djikstra's algorithm
\parencite{Cormen2009algorithms}. However, calculating all possible shortest
paths is not necessary as some routes could never be travelled. Furthermore,
that may be unfeasible in a very large network as the number of paths grows
exponentially. Online calculation of a shortest path between two nodes in a
graph can be efficiently done using \textit{A-star} algorithm by
\textcite{Hart1968paths}.


\subsubsection{Learning}

State is based on the origin, passenger and intended destination. Actions a
taxi can take is enquiring a passenger for destination, offering a price or
driving to another place. If the passenger accepts the price, taxi drives the
passenger to the destination and receives a reward. Keeping track of the exact
passenger is not important. Each time step has a negative reward based on what
action the taxi is taking, always depending on time but also could depend on
distance travelled.

\subsubsection{Software Development}
Ask Nir if this is needed at all?

A paragraph on Kanban? \parencite{Anderson2010kanban}