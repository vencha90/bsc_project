\newpage
\section*{Abstract}
\addcontentsline{toc}{section}{Abstract}
\begin{abstract}~

Taxi fare regulation and fixed taxi fare prices are inefficient. This project
suggests a variable fare pricing approach to improve taxi profitability, where
the prices are chosen using reinforcement learning. Specialised simulation
software called Taxi-sim was developed for this project and published online.
The user of Taxi-sim can specify environmental constraints and market
characteristics, and taxi agents in the simulation are operated by a simple
Q-Learner. The experimental simulation results show that the Q-Learner was able
to improve its profit over time when using variable fare pricing, and that its
profits were significantly larger than when it used fixed fare pricing.
However, the analysis of fixed pricing data showed no profitability and no
profitability improvement over time, either suggesting that the Q-Learner is
not suitable for fixed pricing or revealing an implementation bug. This project
concludes that reinforcement learning can successfully be used for setting
taxi fare prices and justifies further study in the field. Some of the future
work can be done using the Taxi-sim simulation software which has been designed
to be modular and extensible, however it is important to be aware of its
limitations and known bugs.

\end{abstract}
