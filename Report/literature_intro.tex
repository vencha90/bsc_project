\newpage
\section{Background}
\label{sec:literature}

Pricing is one of the main problems in transportation due to its large social
impact, whether it be road tolls, bus prices or regulated taxi fares. One of
the most controversial topics in the taxi market is about whether market
regulation is necessary. There is no consensus on this question, and different
authors give contradictory opinions based both on theoretical studies and
factual real world research. The economics of transportation are covered in
Section \ref{sec:literature:taxis}.

The next part of background that needs covering is artificial intelligence and
reinforcement learning in particular. The various ways of modelling
reinforcement learning problems are investigated in Section
\ref{sec:literature:ai}.

Finally, related works are discussed in Section \ref{sec:literature:related},
spanning both artificial intelligence and transportation economics. A recent
trend with promising results for transport planning and pricing is the usage of
microeconomic models based on individual agents, allowing variable pricing to
be used thus improving system efficiencies. Following this trend software has
been developed to simulate such agent-based scenarios, the most notable of
which is MATSim.
