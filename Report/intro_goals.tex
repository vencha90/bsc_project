\subsection{Goals}
\label{sec:intro:goals}

The project hypothesis is that taxicab profits can be improved by adapting a
dynamic pricing strategy instead of charging fares calculated by a taximeter.
The prices can be set by an AI system using reinforcement learning with
increased profit being the ultimate goal. 

The suggested approach to research this problem is an experimental software
simulation. A reinforcement learning controlled taxi agent travels in some
environment and serves passengers with the aim of maximising its profit. The
environment, passengers and the taxi can be configured as needed for research.

To establish whether the approach worked, it needs to be evaluated. This can be
done by comparing the results with a benchmark of running the simulation with an
identical environment -- this time with fixed fares.
Now a clear set of goals can be extracted from the short project vision
described above:

\begin{itemize}
  \item \textbf{Background.} Research literature and related works to find a
        suitable approach to solving the problem.
  \item \textbf{Software.} Develop a software simulation of a taxi market. 
        The simulation needs to support regulated and unregulated pricing.
  \item \textbf{Data.} Ensure that essential data is gathered and analysed.
        Compare the performance of agents working under price regulation and no
        price regulation to determine whether this project's premise is worth
        any future study.
  \item \textbf{Extensibility.} Develop the software to be modular and
        maintainable in order to be able to simulate various environmental
        conditions, agents and reinforcement learning strategies.
\end{itemize}

Further lower-priority goals that are beneficial to the above core goals can be
identified as well:

\begin{itemize}
  \item \textbf{Reinforcement Learning Strategy.} Compare various reinforcement
        learning strategies,
  \item \textbf{User Interface.} Develop an user-friendly interface.
\end{itemize}
