\subsection{Motivation}
\label{sec:intro:motivation}

Potential to improve public transport effectiveness

Potential to ease the work of regulatory bodies

\subsection{Goals}
\label{sec:intro:goals}

The project hypothesis is that taxicab profits can be improved by adapting a
dynamic pricing strategy instead of charging fares calculated by a taximeter.
The prices can be set by an AI system using reinforcement learning with
increased profit being the ultimate goal. 

\todo{Too many exact details - move to reqs and leave only high level}
This hypothesis can experimentally be tested in a software simulation of a
transport network that consists of: a taxicab (agent) travelling in a network
(graph) between destinations (vertices) that are connected with routes (edges),
where the taxicab can transport a passenger between destinations in return for
a fare. Taxicab offers a fare to a passenger, who in turn can agree or decline
(losing the passenger if declined). Passengers are ready to pay a fare up to a
sum that is function of the journey distance and other  characteristics such as
passenger income and time of day. Thus the taxicab will learn to offer the
right fares for maximised profits depending on environmental circumstances.

To establish whether the approach worked, it needs to be evaluated. This can be
done by comparing the results with a benchmark of running the simulation with an
identical environment -- this time with fixed fares.
Now a clear set of goals can be extracted from the short project vision
described above:

\begin{itemize}
  \item Develop a software simulation of a taxi market. The simulation needs to
        support regulated and unregulated pricing.
  \item Ensure that essential data is gathered and analysed. Compare the 
        performance of agents working under price regulation and no price 
        regulation to determine whether this project's premise is worth any
        future study.
  \item Develop the software to be modular and maintainable to be able to
        simulate various environmental conditions, agents and reinforcement
        learning strategies.
\end{itemize}

Further lower-priority goals that are beneficial to the above core goals can be
identified as well:

\begin{itemize}
  \item Compare various reinfocement leraning strategies
  \item Develop an user-friendly interface
\end{itemize}
