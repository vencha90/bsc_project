\subsection{Goals}

Goals for this project were set in Section \ref{sec:intro:goals}. The main
goals were achieved to a large extent. The lower priority goals were not
achieved, but have good working inroads for future. The goals are discussed in the same order in the paragraphs below.

\textbf{Software.} A taxi market simulation software was successfully developed
and it supports both regulated and unregulated pricing. It is grounded in the
economical theory on  and knowledge about modelling taxi markets. Unfortunately
insufficient attention was paid to making the software to be easily
configurable, therefore there are lots of important default variables that
cannot be transparently changed.

\textbf{Data.} The simulation successfully records its data. It is
relatively easy to change what data is logged. However, data analysis is
somewhat complex and even requires manual data manipulation. To help users with
data analysis, there is an included R language script and instructions on using
it. However, the script will not work out of the box if the logged data is
changed and adds more complexity to manage. Overall, achieving this data
gathering-and-analysis goal required more time and effort than was initially
expected. Nevertheless, the analysis revealed interesting results.

\textbf{Extensibility.} The software implementation is fully covered by tests
and coupling has been reduced to be as low as possible. Maintainability is
ensured by the modularity of software, automated test suite that also serves as
documentation, and the version control commit messages that reveal developer's
intentions for code. The developer is further supported by an extensive
Maintenance Manual (Appendix \ref{sec:maintenance_manual}. Therefore the
software can be easily changed to work with new reinforcement learning
strategies or changed support libraries.

\textbf{Reinforcement Learning Strategy.} This goal was not ever considered due
to time limits. However, it is easily achievable due to software modularity as
was noted above.

\textbf{User Interface.} This goal was only taken in account towards the end of
the project, but never had a priority over the above goals. Nevertheless, the
user interface proved satisfactory in user testing when used in conjunction
with the User Manual (Appendix \ref{sec:user_manual}).


\subsection{Software Development}

In hindsight, the software development process was successful as judged by the
number of achieved goals. Still, some issues could be identified.

The automated test coverage helps to ensure that it the software is working
according to specifications in the tests. However, as was noted in Section
\ref{sec:implementation:testing:automated} it could lead to a false sense of
security, which was proved true later when some unexpected bugs were
discovered. When these bugs were being fixed, a few design oversights were
noticed. This was caused by not revising the system architecture as the project
progressed and uncertainties were being reduced; instead the development was
ongoing and changes in the design were not systematically documented.

Some bugs not picked up by the automated tests were identified only when trying
to use the software. This problem could have been prevented to an extent by
adopting a disciplined approach to automated systems testing similar to one
that was used for unit testing. This would add the difficulty of specifying
system tests which admittedly would have been impossible at uncertain stages of
software development.

While the final software product has low coupling, it had to be reduced
intentionally. Following test driven development at times resulted in a very
isolated unit-based outlook on the system, making it hard to keep the
individual units in line with the whole of architecture. To improve code in
terms of low coupling and other less tangible quality factors, a helpful
practice can be code reviews by others -- unfortunately unavailable to an
useful extent in this project. From a personal point of view, working in a team
is preferable to developing a solo software project as there is limited
feedback one can get when working alone, resulting in a lower quality of work.

The traffic simulation software MATSim was mentioned in Section
\ref{sec:literature:related:simulation}. It can be seen now that using MATSim
could have prevented some of the modelling problems mentioned in this section
and in Section \ref{sec:implementation:software}. However, the performance
constraints identified in Section \ref{sec:implementation:testing:performance}
would likely severely limit the usefulness of using MATSim's micro-grained
approach to traffic modelling, even if no other issues were encountered.


\subsection{Future Work}

The analysis of results (Section \ref{sec:results}) revealed that the project
has been mostly successful and the variable pricing approach could have a
future. Thus it is worth to consider the future outlook of this project. Some
suggestions for developers on fixing known software issues are available in
Maintenance Manual in Appendix \ref{sec:maintenance_manual:future}.

Most importantly, there were some unexpected results showing no improvement and
even worsening performance of reinforcement learning when taxi's prices were
fixed (Section \ref{sec:results:stats}). Before any further work can be done,
the cause of this contradictory result needs to be found and fixed, whether it
is a software bug or a faulty algorithm implementation.

When the above issue is out of the way, further problems with the software were
identified in Section \ref{sec:implementation:software}. These can be fixed in
the order that is deemed necessary for further experiments. 

Even if the smaller issues are not fixed, the simulation can be changed to use
other machine learning algorithms with minimal effort. Of course, more work
would be needed to use machine learning algorithms that are not based on Markov
decision processes.

Extending the simulation to work with multiple taxis is probably the most
interesting future possibility from an economics position. However, the
software would require bigger changes in this case as this change would
implement a multi-agent system.

Passenger modelling was only partially implemented due to the time constraints,
and even having good default values for passenger generation in the simulation
could increase its believability when compared to reality.

If the project is looked at from a less software-centric point of view, there
are other interesting considerations. The literature overview in Section
\ref{sec:literature:taxis} identified taxi regulation as a heated topic in
transportation economics, but very few works touched the subject of the effect
that deregulated prices can have on passengers and the socio-economic
environment. If the variable pricing approach has any future, it needs to be
considered in conjunction with its effects on the society.
