\paragraph{} This section formalises the simulation based on related work in
Section \ref{sec:literature} and applying the results to the context of this
project. Firstly, models for passengers and taxis are described in Section
\ref{sec:design:simulation}. Then Section \ref{sec:design:software} discusses
how these models are going to be implemented as software.

\subsection{Simulation design}
\label{sec:design:simulation}

\paragraph{} The approach suggested by this project is not
compatible with a market where fares are regulated, at least in the current
form of regulation that specifies a formula to calculate fares based on some
variables, usually time and distance. Other ways of regulation that do not
affect pricing, for example, market entry conditions, are compatible with the
suggested variable pricing approach.

\paragraph{}Three different operational types of taxi markets were introduced
in Section \ref{sec:literature:taxis}: phone-order market, cruising taxi market
and taxi rank market. In the phone-order market and taxi rank customers are
actively seeking a taxi, while in the cruising taxi market passengers can only
wait for a taxi to drive by. Therefore the cruising taxi market is chosen as
the easiest target for simulation, extending it to phone-order and taxi rank
markets if the initial experiment is successful.


\subsubsection{Competition}

\paragraph{} The relationship between taxi demand and supply and vice versa was
established in Section \ref{sec:literature:taxis:demand}. This relationship
largely depends on the competitive situation in a taxi market. As the
simulation initially involves only a single taxi, the competition needs to be
replicated by other means. When the simulation is expanded to multiple taxis

\subsubsection{Demand: modelling passengers} 

\paragraph{} Customer demand was reviewed in Section
\ref{sec:literature:taxis:demand} where two approaches to modelling demand were
shown: aggregate and disaggregate. An aggregate demand model "for some portion
of the travel market is a function of variables that describe the product or
its consumers" \parencite{Small2007taxi+urban} The disaggregate approach
specifies a set of variables for each individual passenger. The project goal is
investigating taxi pricing on an individual basis, therefore a disaggregate
approach is preferrable as it allows individual modelling of passengers.
\textbf{review this claim!} Of course, to express demand as a single variable
it needs to be an aggregation of the relevant individual variables.

\paragraph{} Therefore each passenger's demand is a function that of some
variables. The different types of variables affecting taxi demand were
discussed in Section \ref{sec:literature:taxis:demand}. Exogenous demand
variables are value of time and value of reliability that can be derived from a
passenger's income, hour of day when travelling, purpose of travel, social
status, cost of waiting and others; these can be modelled for each individual
passenger. Taxi availability is a variable directly depending on the number of
vacant taxis in an area, this is something that passenger's perceive in
reality. However, for the simulation, taxi availability needs to be assumed a
constant, at least until machine learning capabilities are added to passengers
and a competitive market established beyond the very basic simulation, so that
passangers can learn the availability on their own.

\paragraph{} Let \(P\) be the set of the relevant variables \(p_1,..,p_n\) for
a  passenger: \(P = \{p_1,..,p_n\}\), \newline
where each \(p_i\) has a function \(f_i (p_i) \) that returns a unity-based  
normalised value of \(p_i\), \newline 
and each \(p_i\) has a weight \(w_i\) representing its relative importance  
compared to other variables, where \(\sum_{i=1}^n (w_i) = 1 \). \newline 
Then demand \(Q\) for a taxi ride from origin \(o\) to destination \(d\) at a  
fare price \(F\) can be expressed as: 
\[ Q _{o \rightarrow d} (P,F) = \sum_{i=1}^{n} (w_i \cdot f_i (p_i) )) \]
If \(Q\) is greater than a certain threshold, then the passenger accepts the 
fare, and declines otherwise. This exact threshold \textbf{what will it be?}

\paragraph{} The relevant variables for passengers can be generated using a
stochastic process. Similarly, passenger distribution within the network can be
generated using a stochastic process. These processes can take in account some
characteristics observed in reality such as demand variance during the day,
lower passenger income in some areas resulting in lower willingness to pay, whether a trip is for pleasure or business and others.

\subsubsection{Supply} 

\paragraph{} Let taxis have variable costs \(VC\) consisting of driver's wage
costs \(w\) and fuel expenses \(f\), and a sum of fixed costs \(FC\) including
e.g. deprecation, insurance, business overheads, lease payments. For simplicity
it is assumed that wages are paid for all of the time that a taxi is operating,
and a constant amount of fuel costing \(f\) is used for a unit of distance
covered. Then total taxi costs \(TC\) for a total time \(t\) and an amount
total distance covered \(d\) can be expressed as:
\[ TC = VC + FC = t \cdot w + d \cdot f + FC \]

\subsubsection{Market System}

\paragraph{} The market is represented as a graph constructed of edges (roads)
and vertices (origins/destinations). Each vertex has associated properties: a
list of connected edges, a list of passengers, potentially a list of taxiss.
Each edge has associated properties: length, potentially speed limit, toll
Taxis can travel between vertices using edges. Taxis are assumed to take the
shortest routes and travel at a constant speed. Passengers appear at nodes and
when taxi is at a node it is assumed that they can interact.

\paragraph{} When taxis interact with customers in reality in a market with no
fare regulation, bargaining is likely to happen: taxis bid passengers a fare,
and passengers can agree with it, or decline it and give a countering bid or
abandon the process. Bargaining allows taxis and passengers to agree on a
mutually acceptable price, but has a cost (caused by the time used for
communication) that can be assumed constant for each bid. To simulate real-
world behaviour, the bargaining process will use reinforcement learning as
described in \textcite{Cli1997taxi+bargaining}.

\paragraph{Behaviour: AI}

\subsubsection{Benchmark} 

\paragraph{} To evaluate the variable pricing approach, benchmarks measuring
taxi profitability need to be established using the linear pricing approach.
A market equilibrium fare price can be calculated from the stochastic processes, and a linear tarriff set and a simulation ran with it.

\paragraph{} Comparable simulation runs

\subsubsection{Data} What data will be gathered and how?
