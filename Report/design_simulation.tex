\subsection{Simulation design}
\label{sec:design:simulation}

\subsubsection{Market Considerations}

The approach suggested by this project is not compatible with a market where
fares are regulated, at least in the current form of regulation that specifies
a formula to calculate fares based on some variables, usually time and
distance. Other ways of regulation that do not affect pricing, for example,
market entry conditions, are compatible with the suggested variable pricing
approach.

Three different operational types of taxi markets were introduced in Section
\ref{sec:literature:taxis}: phone-order market, cruising taxi market and taxi
rank market. In the phone-order market and taxi rank customers are actively
seeking a taxi, while in the cruising taxi market passengers can only wait for
a taxi to drive by. Therefore the cruising taxi market is chosen as the easiest
target for simulation, extending it to phone-order and taxi rank markets if the
initial experiment is successful.

The relationship between taxi demand and supply and vice versa was established
in Section \ref{sec:literature:taxis:demand}. This relationship largely depends
on the competitive situation in a taxi market. As the simulation initially
involves only a single taxi, the competition needs to be replicated by other
means. When the simulation is expanded to multiple taxis \textbf{EXTEND}


\subsubsection{Demand: Modelling Passengers}
\label{sec:design:passenger}

Customer demand was reviewed in Section \ref{sec:literature:taxis:demand} where
two approaches to modelling demand were shown: aggregate and disaggregate. An
aggregate demand model "for some portion of the travel market is a function of
variables that describe the product or its consumers"
\parencite{Small2007taxi+urban} The disaggregate approach specifies a set of
variables for each individual passenger. The project goal is investigating taxi
pricing on an individual basis, therefore a disaggregate approach is
preferrable as it allows individual modelling of passengers. \textbf{review
this claim!} Of course, to express demand as a single variable it needs to be
an aggregation of the relevant individual variables.

Therefore each passenger's demand is a function that of some variables. The
different types of variables affecting taxi demand were discussed in Section
\ref{sec:literature:taxis:demand}. Exogenous demand variables are value of time
and value of reliability that can be derived from a passenger's income, hour of
day when travelling, purpose of travel, social status, cost of waiting and
others; these can be modelled for each individual passenger. Taxi availability
is a variable directly depending on the number of vacant taxis in an area, this
is something that passenger's perceive in reality. However, for the simulation,
taxi availability needs to be assumed a constant, at least until machine
learning capabilities are added to passengers and a competitive market
established beyond the very basic simulation, so that passanger agents can
learn the availability on their own.

Let \(P\) be the set of the relevant variables \(p_1,..,p_n\) for
a  passenger: \(P = \{p_1,..,p_n\}\), \newline
where each \(p_i : i \in \{1, .., n\}\) has a function \(f_i (p_i) \) that
returns a unity-based  normalised value of \(p_i\), \newline 
and each \(p_i\) has a weight \(w_i\) representing its relative importance  
compared to other variables, where \(\sum_{i=1}^n (w_i) = 1 \). \newline 
Let \(\delta(o,d)\) be the distance between origin to destination.
Let the passenger's expected fare \(F_{expect}\) depend only on distance

Then demand \(Q\) for a taxi ride from origin \(o\) to destination \(d\) at a  
fare price \(F_{offer}\) can be expressed as: 
\[ Q _{o \rightarrow d} (P,F_{offer}) = \sum_{i=1}^{n} (w_i \cdot f_i (p_i) )) \]

The demand \(Q\) can be used as a probability of a passenger accepting a fare.

The relevant variables for passengers can be generated using a stochastic
process. Similarly, passenger distribution within the network can be generated
using a stochastic process. These processes can take in account some
characteristics observed in reality such as demand variance during the day,
lower passenger income in some areas resulting in lower willingness to pay,
whether a trip is for pleasure or business, and others.


\subsubsection{Taxis in Taxi Market}
\label{sec:design:taxi}

Let taxis have variable costs \(VC\) consisting of driver's wage costs \(w\)
and fuel expenses \(f\), and a sum of fixed costs \(FC\) including e.g.
deprecation, insurance, business overheads, lease payments. For simplicity it
is assumed that wages are paid for all of the time that a taxi is operating,
and a constant amount of fuel costing \(f\) is used for a unit of distance
covered. Then total taxi costs \(TC\) for a total time \(t\) and an amount
total distance covered \(d\) can be expressed as:

\[ TC = VC + FC = t \cdot w + d \cdot f + FC \]

Taxis have a set of available actions. When stopped at a location, they can
decide to drive to another location or wait. If there is a passenger present,
they can start interacting with the passenger, ask for a desired destination
and offer a price.

When taxis interact with customers in reality in a market with no fare
regulation, bargaining is likely to happen: passengers state a destination,
taxis bid passengers a fare, and passengers can agree with it, or decline it
and give a countering bid or abandon the process. Bargaining allows taxis and
passengers to agree on a mutually acceptable price. To simulate real-world
behaviour, a reinforcement learning-based bargaining process can be used as
described in \textcite{Cli1997taxi+bargaining}.

However, sophisticated bargaining can be disregarded in the simulation if the
horizon is significantly long as the agreed fares should converge, albeit at a
slower rate. A simpler approach is limiting the bargaining process to a single
bid which is immediately accepted or declined by the passenger based on the
demand \(Q\). To incentivise the taxi agent, each bid has a cost in time.


\subsubsection{Modelling reinforcement learning}
\label{sec:design:ai}

The research problem needs to be formally defined from a reinforcement learning
point of view according to the definitions in sections \ref{sec:ai:mdp} and
\ref{sec:ai:pomdp}.

It can be assumed that taxi has a complete knowledge of the road network it is
operating in - a realistic assumption given modern GPS navigation systems. In
reality this set would be infinite but it can be simplified to a finite set.
This network forms a part of the total state space. The rest of the state space
is formed of the passenger origin-destination pairs i.e. some state \(s = (o,
d) \). The passenger demand for each of these origin-destination states is
stochastic as mentioned in Section \ref{sec:design:passenger}. 

The actions that a taxi can take were discussed in \ref{sec:design:taxi}. A
simple reward function is as follows. For each moment of time that a taxi is
active, there is an immediate negative reward based on fixed costs and salary
variable costs. If the taxi takes travels, it suffers a negative reward based
on travel variable costs. The only positive reward it gains is from passengers
paying their fares.

If passengers wait at a location for a longer time, then taxis can form
expectations about the possibility of encountering passengers at each location.
To keep track of this uncertainty an agent would need a belief model as
discussed in Section \ref{sec:ai:pomdp}. This complexity can be avoided by
assuming that passengers do not wait for taxis and appear at a location for a
single moment in time only.


\subsubsection{Fixing Stochastic Variability}

\subsubsection{Benchmark} 

To evaluate the variable pricing approach, benchmarks measuring taxi
profitability need to be established using the linear pricing approach. Market
equilibrium demand and fare prices can be calculated from the stochastic
processes, and a linear tarriff set and a simulation ran with it. It will
establish the average best-case scenario for both passengers and taxis. If the
reinforcement learning agent can approach or exceed similar profits it will
mean that the experiment was a success. A simple example of this follows based on sections \ref{sec:design:taxi} and \ref{sec:design:passenger}.

We assume that a passenger's yearly income is their only determinant of demand
and that the yearly income for all passengers is linearly distributed with an
average of £25,000. Then a passenger \(P = {p_1} \) where income \(p_1 = £
20,000\), and the unity- based normalising function \( f_1(p_1) \) will return
0.4. HEREHERE Therefore

\subsubsection{Data}

It is possible to record rewards for each step of time. This data will show
overall profits and reveal how quickly profitability was reached, if at all.
