\newpage
\section{Maintenance Manual}
\label{sec:maintenance_manual}

The Maintenance Manual provides details of the software implementation. It
gives an advanced overview of the system setup and installation and lists the
main dependencies in Section \ref{sec:maintenance_manual:installation}. A
detailed guide to developing the software is given in Section
\ref{sec:maintenance_manual:developing}, including testing, documentation and
suggested areas of further development.


 To readers unfamiliar with Ruby, it is important to know that the term
\textit{gems} are Ruby software packages. The project is available online at
this address: \url{https://github.com/vencha90/taxi-sim}. You might also have
received a \texttt{tar} archive file of the software.

\subsection{Installation of Development Environment}
\label{sec:maintenance_manual:installation}

All commands in this section are supposed to be ran on your main operating
system.


\subsubsection{Installation Without Virtualisation}
\label{sec:maintenance_manual:native_install}

It is recommended to use virtualisation as described in User Manual in Appendix
\ref{sec:user_manual:installation:virtualisation}. Virtualisation will ensure
that only the correct required dependencies are installed and your operating
system is not corrupted by potentially conflicting software. If you still do
not wish to use virtualisation, there are alternatives for Unix-like systems.

You can install Ruby Version Manager (RVM) \parencite{Rvm} from
\url{http://rvm.io/} and then install Ruby by running \texttt{rvm install
2.1.1} in terminal. You also need to install Bundler by
running \texttt{gem install bundler \&\& bundle install --local} to manage
dependencies.

Instead of RVM you could use \texttt{rbenv} \parencite{Rbenv}. Unfortunately
RVM and rbenv are mutually exclusive and could cause conflicts with other parts
of an OS. The worst option is installing Ruby completely natively as this
provides no isolation whatsoever from other OS components.


\subsubsection{Developer Setup}

Even if you have the source code, you should use the latest version-controlled
software from the official code repository at \url{https://github.com/vencha90
/taxi-sim}.

To acquire the source code for development, please follow the instructions
below. Commands that should be used in command line terminal are marked
\textit{(cmd)}, commands that require downloading software from the world wide
web are marked \textit{(web)}.

\begin{itemize}
  \item \textit{(web)} Install Git \parencite{Git} version control from
        \url{http://git-scm.com/}.
  \item \textit{(cmd)} Open a command line terminal.
  \item \textit{(cmd)} Clone source code to a suitable directory on your local
        hard drive using this command:
        \texttt{git clone git@github.com:vencha90/taxi-sim.git}
  \item \textit{(cmd)} Change the working directory to the project's source
        code: \texttt{cd taxi-sim}.
\end{itemize}


\subsubsection{Dependencies}

Top-level gems are listed in \texttt{Gemfile} and explained below. Some gems
are grouped together and labelled as for development and testing, meaning that
they are not necessary for simply using the program. Potentially these labels
could be used to release two separate software versions -- standard and
developer. The full list of dependencies is listed in \texttt{Gemfile.lock} and
not explained as they only play a supporting role.

\texttt{require\_all} \parencite{Require+all} eases source code management by
allows to require all source files in a directory on a single line in an OS-
agnostic way (not supported by Ruby natively).

\texttt{plexus} \parencite{Plexus} is a graphing gem. It is discussed in more
detail in Section \ref{sec:implementation:software:roads}

\texttt{thread\_safe, '0.3.1'} is actually a dependency for other top-level
gems, but was listed explicitly to specify a compatible release version
(0.3.1).

\texttt{guard} and \texttt{guard-rspec} \parencite{Guard} can be used for
monitoring file system events. A recommended work flow is suggested in Section
\ref{sec:maintenance_manual:testing}

\texttt{rake} \parencite{Rake} is a common tool used for Ruby command line
applications.

\texttt{simplecov, '~> 0.7.1'} \parencite{Simplecov} monitors code test
coverage as discussed in Sections \ref{sec:design:software:tdd} and
\ref{sec:implementation:testing:automated}. The version number is specified due
to a bug in the latest version.

\texttt{rspec} \parencite{Rspec} is a testing framework. It is discussed in
Sections \ref{sec:design:software:tdd} and
\ref{sec:implementation:testing:automated}. Development with Rspec is discussed
in Section \ref{sec:maintenance_manual:testing}.


\subsection{Developer's Guide}
\label{sec:maintenance_manual:developing}

All commands in this section are supposed to be ran from a terminal in your
development environment. This section explains how to work with the automated
tests, where to find documentation and how to customise the software. Class and
module names are used in this section, please see Section
\ref{sec:maintenance_manual:documentation} for reference.


\subsubsection{Tests}

\label{sec:maintenance_manual:testing}

Test-Driven Development (TDD) was followed as much as was reasonable when
developing this project, and you are recommended to do the same. Please see
Sections \ref{sec:design:tdd} and \ref{sec:implementation:testing:automated} to
find out more about TDD.

Documentation and support is readily available online for Rspec. An example
Rspec scenario was shown in Figure \ref{figure:rspec}, but the simplistic
language does not limit the functionality of the framework. It can even be
easily extended if needed.

All Rspec scenarios can be run by entering \texttt{bundle exec rspec} in
terminal, and help displayed by entering \texttt{bundle exec rspec -h}.

A typical developer's TDD work flow with Rspec would go as follows: write a test
in text editor, switch context to command line, run the test in question
(because running all tests would take a lot of time in a bigger codebase),
switch context back...

This process can be automated by Guard, ordering it to monitor changes in test
files and the linked code files. Guard can also be assigned conditions when the
full test suite needs to be ran. It has already been configured for this
project and can be started by entering \texttt{bundle exec guard}.
Unfortunately this functionality is not available on virtual machines running
on Windows. For virtual machines on other OSs, Guard and Vagrant needs
configuration as shown here: \url{https://github.com/guard/guard#-o--listen-on-
option}

All tests are located in \texttt{specs/} directory. Please see Section
\ref{sec:maintenance_manual:documentation} for an overview of files and
locations.


\subsubsection{Documentation}
\label{sec:maintenance_manual:documentation}

Documentation for this software is in two forms. Table
\ref{table:software:classes} in this section provides an overview of classes
and modules with their related test files, and Table
\ref{table:software:support} lists the remainder of noteworthy files. Secondly,
detailed documentation for each class is the automated Rspec tests.
Additionally, Git commit history may prove useful for understanding the
developer's intentions at the time of writing code.

\newcommand{\specialcell}[2][c]{%
  \begin{tabular}[#1]{@{}l@{}}#2\end{tabular}}
\begin{table}
\begin{tabular}{| p{0.3 \textwidth} | p{0.6 \textwidth} |}
  \hline
  lib/  & Directory for source code files \\ \hline
  spec/ & Directory for Rspec test files \\ \hline
  \specialcell{lib/file\_parser.rb \\ spec/file\_parser\_spec.rb} & Input file processing -- \texttt{FileParser} class \\ \hline
  \specialcell{lib/graph/ \\ spec/graph/ } & Road Network -- \texttt{Graph} and \texttt{Graph::Vertex} classes \\ \hline
  lib/logging.rb & Logging module -- \texttt{Logging} \\ \hline
  \specialcell{lib/passenger/ \\ spec/passenger/} & Passenger -- \texttt{Passenger} and \texttt{Passenger::Characteristic} classes \\ \hline
  \specialcell{lib/taxi/ \\ spec/taxi/} & Taxi representation, including the Q-Learner -- \texttt{Taxi}, \texttt{Taxi::Action}, \texttt{Taxi::State} and \texttt{Taxi::Learner} classes \\ \hline
  \specialcell{lib/taxi\_learner.rb \\ spec/taxi\_learner\_spec.rb} & Main class and top-level namespace for the simulation -- \texttt{Runner} \\ \hline
  \specialcell{lib/world.rb \\ spec/world\_spec.rb} & Environment -- \texttt{World} class. \\ \hline
\end{tabular}
\caption{
  Core Software Files
  \label{table:software:classes}
}
\end{table}

\begin{table}
\begin{tabular}{| p{0.3 \textwidth} | p{0.6 \textwidth} |}
  \hline
  analysis.R & R language script for data analysis \\ \hline
  bin/taxi\_learner & Executable file for shell \\ \hline
  .gitignore & File list for Git to ignore from version control \\ \hline
  spec/fixtures/ & Fixture files for automated tests \\ \hline
  Guardfile & Specifications for Guard runtime \\ \hline
  \specialcell{Gemfile \\ Gemfile.lock} & Bundler gem specifications \\ \hline
  logs/ & Log files from simulation -- \texttt{simulation.log} and \texttt{summary.log} \\ \hline
  Rakefile & Specifications for Rake tasks \\ \hline
  \specialcell{.rspec \\ spec/spec\_helper.rb}  & Rspec configuration and support \\ \hline
  \specialcell{Vagrantfile \\ vagrant\_manifest.pp} & Vagrant configuration \\ \hline
  .vagrant & Vagrant-created files for virtualisation \\ \hline
  vendor/ & Gems stored for offline usage of the software \\ \hline
\end{tabular}
\caption{
  Supporting Software Files
  \label{table:software:support}
}
\end{table}


\subsubsection{Customising Output}
\label{sec:maintenance_manual:customising_output}

Customised log details is one of the simplest changes to make to the system. It
will not break automated tests if changes are made only to the output
parameters, as tests only check that \texttt{any} output is produced.

Logs are currently being written from World, Taxi and Passenger classes and are
handled by Logger module. To change the variables being written, you simply
have to include them in \texttt{log\_params} method.

Logger module can easily be included in other classes as needed, and variables
to output can be specified just like in World, Taxi or Passenger classes.


\subsubsection{Future Work}
\label{sec:maintenance_manual:future}

Some issues with the software were identified in Section
\ref{sec:implementation:software:issues}. This section will give suggest how
they could be fixed.

To transparently manage default values, they can be stored in a single
\texttt{.yml} configuration file and retrieved when necessary. YAML parsing is
already included in the system in FileParser class, so the hardest part of this
task would be referencing the old variables to the new file. It is easier to
identify the old variables, as they are currently stored as ALL\_CAPS constants
in source code files.

Fixing the taxi action sequence is aided by automated tests. However, this task
will require serious refactoring as it would likely change the Taxi class
significantly, and may need to change its subclasses as well.

Probably the easiest issue to fix is providing a progress indicator to users.
This can easily be done by Runner or World classes that have access to the
global time.
