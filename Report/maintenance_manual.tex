\clearpage
\section{Maintenance Manual}
\label{sec:maintenance_manual}

(5, 500-1000 w?)

Is there a description of the system and any third-party software required
including where to get it from?

Are there installation instructions and how good are they?

Is there a list of packages and files and what they are for?

Is there a description of how the system can be changed for most likely future
adaptations and extensions?

\todo{remove notes}

\todo{More specific references to maintenance manual from the report}

The Maintenance Manual provides details of the software implementation. To
readers unfamiliar with Ruby, it is important to know that the term
\textit{gems} are Ruby software packages.

The project is available online at this address:
\url{https://github.com/vencha90/taxi-sim}. You might also have received a
\textit{tar} archive file of the software.

\todo{introduction of contents}


\subsection{Installation of Development Environment}

All commands in this section are supposed to be ran on your main operating
system.

\subsubsection{System Setup}

It is recommended to use virtualisation as described in User Manual in Appendix
\ref{sec:user_manual:installation:virtualisation}.


\subsubsection{Installation Without Virtualisation}
\label{sec:maintenance_manual:native_install}

Virtualisation will ensure that only the correct required dependencies are
installed and your operating system is not corrupted by potentially conflicting
software. If you still do not wish to use virtualisation, there are
alternatives for unix-like systems.

You can install Ruby Version Manager (RVM) \parencite{Rvm} from
\url{http://rvm.io/} and then install Ruby by running \textit{rvm install
2.1.1} in terminal. You also need to install Bundler by
running \textit{gem install bundler \&\& bundle install --local} to manage
dependencies.

Instead of RVM you could use \textit{rbenv} \parencite{Rbenv}. Unfortunately
RVM and rbenv are mutually exclusive and could cause conflicts with other parts
of an OS. The worst option is installing Ruby completely natively as this
provides no isolation whatsoever from other OS components.


\subsubsection{Developer Setup}

Even if you have the source code, you should use the latest version-controlled
software from the official code repository at \url{https://github.com/vencha90
/taxi-sim}.

To acquire the source code for development, please follow the instructions
below. Commands that should be used in command line terminal are marked
\textit{(cmd)}, commands that require downloading software from the world wide
web are marked \textit{(web)}.

\begin{itemize}
  \item \textit{(web)} Install Git \parencite{Git} version control from
        \url{http://git-scm.com/}.
  \item \textit{(cmd)} Open a command line terminal.
  \item \textit{(cmd)} Clone source code to a suitable directory on your local
        hard drive using this command:
        \textit{git clone git@github.com:vencha90/taxi-sim.git}
  \item \textit{(cmd)} Change the working directory to the project's source
        code: \textit{cd taxi-sim}.
\end{itemize}


\subsubsection{Dependencies}

Top-level gems are listed in \textit{Gemfile} and explained below. Some gems
are grouped together and labelled as for development and testing, meaning that
they are not necessary for simply using the program. Potentially these labels
could be used to release two separate software versions -- standard and
developer. The full list of dependencies is listed in \textit{Gemfile.lock} and
not explained as they only play a supporting role.

\textit{require\_all} \parencite{Require+all} eases source code managament by
allows to require all source files in a directory on a single line in an OS-
agnostic way (not supported by Ruby natively).

\textit{plexus} \parencite{Plexus} is a graphing gem. It is discussed in more
detail in Section ref{sec:design:network}

\textit{thread\_safe, '0.3.1'} is actually a dependency for other top-level
gems, but was listed explicitly to specify a compatible release version
(0.3.1).

\textit{guard} and \textit{guard-rspec} \parencite{Guard} can be used for
monitoring filesystem events. A recommended workflow is suggested in Section
\ref{sec:maintenance_manual:testing}

\textit{rake} \parencite{Rake} is a common tool used for Ruby command line
applications.

\textit{simplecov, '~> 0.7.1'} \parencite{Simplecov} monitors code test
coverage as discussed in Sections \ref{sec:design:software:tdd} and
\ref{sec:implementation:testing:automated}. The version number is specified due
to a bug in the latest version.

\textit{rspec} \parencite{Rspec} is a testing framework. It is discussed in
Sections \ref{sec:design:software:tdd} and
\ref{sec:implementation:testing:automated}. Development with Rspec is discussed
in Section \ref{sec:maintenance_manual:testing}.


\subsection{Developer's Guide}
\label{sec:maintenance_manual:testing}

All commands in this section are supposed to be ran from a terminal in your
development environment.

Test-Driven Development (TDD) was followed as much as was reasonable when
developing this project, and you are recommended to do the same. Please see
Sections \ref{sec:design:tdd} and \ref{sec:implementation:testing:automated} to
find out more about TDD.

Documentation and support is readily available online for Rspec. An example
Rspec scenario was shown in Figure \ref{figure:rspec}, but the simplistic
language does not limit the functionality of the framework. It can even be
easily extended if needed.

All Rspec scenarios can be run by entering \textit{bundle exec rspec} in
terminal, and help displayed by entering \textit{bundle exec rspec -h}.

A typical developer's TDD workflow with Rspec would go as follows: write a test
in text editor, switch context to command line, run the test in question
(because running all tests would take a lot of time in a bigger codebase),
switch context back...

This process can be automated by Guard, ordering it to monitor changes in test
files and the linked code files. Guard can also be assigned conditions when the
full test suite needs to be ran. It has already been configured for this
project and can be started by entering \textit{bundle exec guard}.
Unfortunately this functionality is not available on virtual machines running
on Windows. For virtual machines on other OSs, Guard and Vagrant needs
configuration as shown here: \url{https://github.com/guard/guard#-o--listen-on-
option}

All tests are located in \textit{specs/} directory. Please see Section
\ref{sec:maintenance_manual:documentation} for an overview of files and
locations.


\subsubsection{Customising Output}
\label{sec:maintenance_manual:customising_output}

Customised log details is one of the simplest changes to make to the system. It
will not break tests if changes are made only to the output parameters, as
tests only check that any output is produced.

To customise the output parameters for any of taxi, world or 





\subsection{Documentation}

\subsubsection{Automated Tests}
\todo{rspec flow} rspec guard etc

\subsubsection{List of Files}

\todo{List of files}
table
code \& spec files | description
