\subsection{Overview}
\label{sec:intro:overview}

This is a report on an experimental simulation in machine learning and taxi
economics. It was written in support of an undergraduate project for BSc
Computing Science degree at University of Aberdeen.

Taxis (in some literature the term \textit{taxicab} is used) are an important
part of transportation worldwide and usually are regulated by local
governments. However, fare regulation creates economic inefficiencies because
the demand and supply greatly differs based on many different factors.
Therefore no regulation could be preferable when the taxi's fare price is set
freely. This project introduces Taxi-sim, a software for taxi market simulation
where Q-Learning is used to set taxi's fare prices.

The report starts by introducing main and optional goals for the project in
Section \ref{sec:intro:goals}. Detailed background information is given in
Section \ref{sec:literature}, investigating transportation economics,
reinforcement learning and related works. The core requirements for a software
implementation are established in Section \ref{sec:requirements}. Section
\ref{sec:design} describes the chosen software development approach as well as
the architecture and technologies that are used. The finished software is
discussed in sections on implementation (\ref{sec:implementation:software}) and
testing (\ref{sec:implementation:testing}). The simulation experiment and the
overall success of the project is discussed in Section \ref{sec:evaluation}.
Finally, the project is summed up in Section \ref{sec:conclusion}.

The project has a homepage which also serves as the main web page for the
source code repository, located online at this address:
\url{https://github.com/vencha90/taxi- sim}. Please use the project homepage to
find the most up-to-date URLs if any of the URLs specified in the report are
not working. The results of the experiment are available online at this address
\url{https://db.tt/zYPt09CL}.
