\paragraph{}
This is where it all starts.

\subsection{Taxis}
Taxis (also known as \textit{taxicabs}) are an important part of public transportation. Because of their prevalence worldwide and importance in transportation a wide range of literature has been produced on taxis. For this project, the most relevant area of this literature is economical modelling of taxi markets, an overview of which is given by \textcite{Salanova2011taxi+review}. A major topic in the research and discussion on taxis is taxi market regulation, and parts of it are relevant and will be considered in some detail. Three different taxi markets can be distinguished: cruising taxi market when a passenger hails the taxi no the street, phone-order taxi market, and taxi ranks where multiple taxis wait for passengers.
\paragraph{Taxi Modelling} was first done by \textcite{Douglas1972taxi+regulation} according to \textcite{Salanova2011taxi+review}.  He investigated a regulated cruising-taxi market (where a customer hails a taxi on the street on visual contact) and defined the fundamental taxi problem to be finding an equilibrum of an optimal level of service matching an optimal price. His limited model has been used as reference by all the later authors cited by \textcite{Salanova2011taxi+review} that have extended it to other taxi markets and factored in more environmental influences. 
The most recent publications are sophisticated models by \textcite{Wong2008taxi+modeling}, \textcite{Yang2010taxi+equilibria}, \textcite{Yang2010taxi+nonlinear}. These 
\paragraph{Regulation?????} is a sensitive topic for taxi research as no general consensus has been reached on whether it is recommended. \textcite{Oecd2007taxi+policy}, cited by \parencite{Salanova2011taxi+review} lists arguments both for and against regulation as observed in different countries. The approach suggested by this project is not compatible with a market where fares are regulated, at least in the current form of regulation that specifies a formula to calculate fares based on some variables, usually time and distance. Other ways of regulation, for example, entry conditions, are compatible with the suggested approach and are not investigated any further. It is important to note that some markets considered \textit{deregulated} still have some form of fare regulation, for example, taxis in New Zealand are required to list their maximum fares based on time and distance \parencite{Gaunt1995taxi+newzealand}, but are not forced to follow them.

\subsection{Reinforcement learning}