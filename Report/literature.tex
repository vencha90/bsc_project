\paragraph{}
This is where it all starts.

\subsection{Taxis}
Taxis (also known as \textit{taxicabs}) are an important part of public transportation. Because of their prevalence worldwide and importance in transportation a wide range of literature has been produced on taxis. For this project, the most relevant area of this literature is economical modelling of taxi markets, an overview of which is given by \textcite{Salanova2011taxi+review}. A major topic in the research and discussion on taxis is taxi market regulation, and parts of it are relevant and will be considered in some detail. Three different taxi markets can be distinguished: cruising taxi market when a passenger hails the taxi no the street, phone-order taxi market, and taxi ranks where multiple taxis wait for passengers.
\paragraph{}Regulation is a sensitive topic for taxi research as no general consensus has been reached on whether it is recommended. \textcite{Cairns1996taxi+competition} investigated economic workings of taxi markets and incorporated results of earlier research in their economic equilibria findings. They concluded that regulation is needed to achieve non-negative profits (the so-called economic second best). \textcite{Oecd2007taxi+policy}, cited by \textcite{Salanova2011taxi+review} lists arguments both for and against regulation as observed in different countries, and notes that markets with widely varying regulation can operate successfully. SHOULD THIS BE HERE? ==>> The approach suggested by this project is not compatible with a market where fares are regulated, at least in the current form of regulation that specifies a formula to calculate fares based on some variables, usually time and distance. Other ways of regulation, for example, entry conditions, are compatible with the suggested approach and do not require any further investigation. It is important to note that some markets considered \textit{deregulated} still have some form of fare regulation, for example, taxis in New Zealand are required to list their maximum fares based on time and distance \parencite{Gaunt1995taxi+newzealand}, but are not forced to follow them.
\paragraph{}Taxi market modelling was first done by \textcite{Douglas1972taxi+regulation} according to \textcite{Salanova2011taxi+review}. He investigated a regulated cruising-taxi market (where a customer hails a taxi on the street on visual contact) and defined the fundamental taxi problem to be finding an equilibrum of an optimal level of service matching an optimal price. His limited model has been used as reference by all the later authors cited by \textcite{Salanova2011taxi+review} that have extended it to other taxi markets and factored in more environmental influences. \textcite{Devany1975taxi+capacity} researched regulated taxi markets organised as a franchised monopoly, using a medallion system, and having free entry, having the goal of finding equilibrium output, capacity and utilisation. He gave a formula tor passenger demand depending on taxi fare, passenger value of time and waiting time. \textcite{Manski1967taxi+demand} analyse the taxi market from a purely economical point of view and conclude that in addition to exogenous variables, passenger demand for taxi services is also directly related to taxi supply through waiting time. Similarly, taxi supply is influenced by taxi utilisation, which in turn directly depends on passenger demand.
\paragraph{}The most recent publications are sophisticated models based on the network model for cruising-taxi market by \textcite{Yang1998taxi+network}. This network was modelled as a graph and assumed constant taxi demand and supply, passenger demand was represented as origin-destination matrices. Finally, this paper suggested an algorithm to find an equilibrum for the optimal number of taxis in a market and equations to calculate taxi utilisation and customer waiting time. In contrast, \textcite{Yang2000taxi+utilization} focuses on supply and demand to recommend optimal policies for taxi regulation in Hong Kong and base their model on various data sources. A number of exogenous and endogenous variables affecting taxi market are identified, and equations are suggested to calculate them: passenger waiting time, percentage of occupied taxis, vacant taxi headway, daily taxi passenger trips and taxi waiting time. This model can be used to forecast taxi demand, taxi utilization and service quality, although the authors warn that it does not take in account the complex supply-demand relationships in taxi market.
\paragraph{}Consequently \textcite{Yang2002taxi+demand} continued to evaluate the supply-demand equilibria of taxi market started by \textcite{Yang1998taxi+network} and \textcite{Yang2000taxi+utilization}, resulting in the conclusion that the spatial characteristics of a network where taxis are operating strongly influence supply, demand and should bear weight when evaluating regulatory policies.This study focuses on social surplus (the sum of customer surplus and producer surplus) as the key objective of taxi markets. Four different regulatory frameworks that could be applied to taxi markets are investigated: free entry and unconstrained fare, free entry and regulated fare, regulated entry and unconstrained fare, and regulated entry and regulated fare. All of these cases are investigated with both competitive and monopilistic markets, and equilibria are found.
\paragraph{}
\textcite{Wong2008taxi+modeling} extended this model to heterogenous vehicle and user classes, and included congestion which is a major issue in reality particularly in big cities, but was ignored by most of earlier research. The way how cruising taxis and customers find each other was researched by \textcite{Yang2010taxi+equilibria}, paying particular attention to customer behaviour. \textcite{Yang2010taxi+nonlinear} proposes a nonlinear fare structure to correct market and regulatory inefficiencies, and applies it to a similar model.


\subsection{Reinforcement learning}